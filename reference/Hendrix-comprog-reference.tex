% -*- compile-command: "pdflatex -shell-escape Hendrix-comprog-reference.tex" -*-

\documentclass[10pt]{book}

\usepackage[margin=1in]{geometry}
\usepackage{fancyhdr}

\pagestyle{fancy}

% \lhead{\leftmark}
% \rhead{\rightmark}

% \lfoot{Hendrix Programming Team Reference}
% \rfoot{\thepage}

% TODO: make fancy header with page number and section etc.

\usepackage[T1]{fontenc}
\usepackage[utf8]{inputenc}
\usepackage{hyperref}

\usepackage{graphicx}
\graphicspath{{images/}}

\usepackage{minted}
\newcommand{\code}[1]{\inputminted[fontsize=\normalsize]{java}{code/#1}}
\newcommand*{\fulllink}[1]{\hyperref[{#1}]{\nameref*{#1}~(\ref*{#1}, page~\pageref*{#1})}}
\newcommand*{\link}[1]{\hyperref[{#1}]{(\ref*{#1}, page~\pageref*{#1})}}
\newcommand{\kattis}[1]{\href{https://open.kattis.com/problems/#1}{\texttt{#1}}}

\newcommand{\todo}[1]{\textcolor{red}{[TODO: #1]}}

\begin{document}

\title{Hendrix Programming Team Reference}

\maketitle

\tableofcontents
\newpage

\chapter{Limits}

\begin{itemize}
\item Rule of thumb: $10^8$ operations per second.
  \todo{table of runtimes, discussion of brute force}
\item $2^{10} = 1024 \approx 10^3$
\item $2 \cdot 10^9$ fits in a 32-bit \texttt{int}.
\item $9 \cdot 10^{18}$ fits in a 64-bit \texttt{long}.  Remember to
  use \texttt{long} if you need an answer $\bmod (10^9 + 7)$ (which
  would fit in an \texttt{int}) but computing the answer requires
  \emph{multiplying} $\bmod (10^9 + 7)$.
\end{itemize}

\chapter{Java Reference}

\section{Template}

\code{java/Template.java}

\section{Scanner}

\texttt{Scanner} is relatively slow but should usually be sufficient
for most purposes.  If the input or output is relatively large (> 1MB)
and you suspect the time taken to read or write it may be a hindrance,
you can use \fulllink{sec:fastio}.

\code{java/ScannerExample.java}

\section{String}

\section{Arrays}

\section{ArrayList}

\section{Stack}

\section{Queue}

ArrayDeque

\section{PriorityQueue}

\section{Set}

HashSet, TreeSet

\section{Map}

HashMap, TreeMap

\section{BigInteger}

\section{Sorting}

\todo{Basic template for implementing Comparable}
\todo{Arrays.sort, Collections.sort}

\section{Fast I/O} \label{sec:fastio}

\todo{Kattio}

\chapter{Data Structures}

\section{Union-find}

A union-find structure can be used to keep track of a collection of
disjoint sets, with the ability to quickly test whether two items are
in the same set, and to quickly union two given sets into one.  It is
used in Kruskal's Minimum Spanning Tree algorithm \link{sec:kruskal},
and can also be useful on its own.  \texttt{find} and \texttt{union}
both take essentially constant amortized time.

\includegraphics[height=0.9\baselineskip]{Kattis} \kattis{10kindsofpeople}, \kattis{drivingrange},
\kattis{islandhopping}, \kattis{kastenlauf}, \kattis{lostmap},
\kattis{minspantree}, \kattis{numbersetseasy}, \kattis{treehouses},
\kattis{unionfind}, \kattis{virtualfriends},
\kattis{wheresmyinternet},

\code{data-structures/UnionFind.java}

\section{Segment tree}

\section{Fenwick tree}

\section{Heap}

\section{Red-black tree}

\chapter{Graphs}

\section{Representation}

  Adjacency matrix, adjacency maps.  Edge objects.

  Implicit graphs.

\section{Traversal (DFS/BFS)}

Code for DFS/BFS with level labelling, parent map.

\section{Single-source shortest paths (Dijkstra)}

\section{All-pairs shortest paths (Floyd-Warshall)}

\section{Min spanning tree (Kruskal)} \label{sec:kruskal}

\section{DAGs, topological sort}

\section{Max flow}

\section{Miscellaneous}

\todo{New virtual source/sink node trick}

\chapter{Dynamic Programming}

\chapter{Strings}

\section{Suffix array}

\chapter{Divide \& Conquer}

\section{Counting inversions}

\chapter{Mathematics}

\section{GCD/Euclidean Algorithm}

\section{Fractions}

\section{Primes and factorization}

\todo{Basic primality testing and factorization with trial division.
  Sieving (primes, factors, Euler totient).}

\section{Combinatorics}

\todo{Basic principles of combinatorics.  Code for computing binomial
  coefficients.}

\chapter{Bit Tricks}

\todo{Basic bit manipulation.  Using bitstrings to compactly represent
  sets/states.  Iterating through all subsets with counter.}

\chapter{Geometry}

\chapter{Miscellaneous}

\section{2D grids}

\todo{Discussion of implicit graphs}
\todo{Formulas for converting between pair of coordinates and single index}
\todo{Trick for listing neighbors with delta vector}

\chapter{Python reference}

\todo{Basic Python template for doing I/O etc.}
\todo{Benefits of Python (big integer, dictionaries)}

\end{document}
