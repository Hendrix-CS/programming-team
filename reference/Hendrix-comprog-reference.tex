% -*- compile-command: "pdflatex -shell-escape Hendrix-comprog-reference.tex" -*-

\documentclass[10pt]{article}

\usepackage[noheadfoot,margin=1in]{geometry}

% TODO: make table of contents
% TODO: make fancy header with page number and section etc.

\usepackage[T1]{fontenc}
\usepackage[utf8]{inputenc}
\usepackage{hyperref}

\usepackage{minted}
\newcommand{\code}[1]{\inputminted[fontsize=\normalsize]{java}{code/#1}}

\newcommand{\kattis}[1]{\href{https://open.kattis.com/problems/#1}{\texttt{#1}}}

\begin{document}

\title{Hendrix Programming Team Reference}

\maketitle

\tableofcontents
\newpage

\section{Limits}

\begin{itemize}
\item Rule of thumb: $10^8$ operations per second.
  XXX table of runtimes, discussion of brute force
\item $2^{10} = 1024 \approx 10^3$
\item $2 \cdot 10^9$ fits in a 32-bit \texttt{int}.
\item $9 \cdot 10^{18}$ fits in a 64-bit \texttt{long}.  Remember to
  use \texttt{long} if you need an answer $\bmod (10^9 + 7)$ (which
  would fit in an \texttt{int}) but computing the answer requires
  \emph{multiplying} $\bmod (10^9 + 7)$.
\end{itemize}

\section{Java Reference}

\subsection{Template}

\code{java/Template.java}

\subsection{Scanner}

\texttt{Scanner} is relatively slow but should usually be sufficient
for most purposes.  If the input or output is relatively large (> 1MB)
and you suspect the time taken to read or write it may be a hindrance,
you can use \hyperref[sec:fastio]{Fast IO}.

\code{java/ScannerExample.java}

\subsection{String}

\subsection{Arrays}

\subsection{ArrayList}

\subsection{Stack}

\subsection{Queue}

ArrayDeque

\subsection{Set}

HashSet, TreeSet

\subsection{Map}

HashMap, TreeMap

\subsection{BigInteger}

\subsection{Sorting}

\subsection{Fast IO} \label{sec:fastio}

\section{Data Structures}

\subsection{Union-find}

A union-find structure can be used to keep track of a collection of
disjoint sets, with the ability to quickly test whether two items are
in the same set, and to quickly union two given sets into one.  It is
used in Kruskal's Minimum Spanning Tree algorithm, and can also be
useful on its own.  \texttt{find} (with path compression) and
\texttt{union} both take essentially constant amortized
time.

Kattis: \kattis{10kindsofpeople}, \kattis{drivingrange},
\kattis{islandhopping}, \kattis{kastenlauf}, \kattis{lostmap},
\kattis{minspantree}, \kattis{numbersetseasy}, \kattis{treehouses},
\kattis{unionfind}, \kattis{virtualfriends},
\kattis{wheresmyinternet},

\code{data-structures/UnionFind.java}

\subsection{Segment tree}

\subsection{Fenwick tree}


\section{Graphs}

\subsection{Traversal (DFS/BFS)}

\subsection{Single-source shortest paths (Dijkstra)}

\subsection{All-pairs shortest paths (Floyd-Warshall)}

\subsection{Min spanning tree (Kruskal)} \label{sec:kruskal}

\subsection{Topological sort}

\subsection{Max flow}

\section{Dynamic Programming}

\section{Strings}

\subsection{Suffix array}

\section{Divide \& Conquer}

\subsection{Counting inversions}

\section{Mathematics}

\subsection{Fractions}

\subsection{Binomial coefficients}

\subsection{GCD/Euclidean Algorithm}

\subsection{Primality}

\section{Geometry}

\end{document}
