% -*- compile-command: "pdflatex -shell-escape Hendrix-comprog-reference.tex" -*-

\documentclass{article}

\usepackage[T1]{fontenc}
\usepackage[utf8]{inputenc}
\usepackage{hyperref}

\usepackage{minted}
\newcommand{\code}[1]{\inputminted[fontsize=\normalsize]{java}{code/#1}}

\newcommand{\kattis}[1]{\href{https://open.kattis.com/problems/#1}{\texttt{#1}}}

\begin{document}

\section{Data Structures}

\subsection{Union-find}

A union-find structure can be used to keep track of a collection of
disjoint sets, with the ability to quickly test whether two items are
in the same set, and to quickly union two given sets into one.  It is
used in Kruskal's Minimum Spanning Tree algorithm, and can also be
useful on its own.

Kattis: \kattis{10kindsofpeople}, \kattis{drivingrange},
\kattis{islandhopping}, \kattis{kastenlauf}, \kattis{lostmap},
\kattis{minspantree}, \kattis{numbersetseasy}, \kattis{treehouses},
\kattis{unionfind}, \kattis{virtualfriends},
\kattis{wheresmyinternet},

\code{data-structures/UnionFind.java}

\section{Mathematics}

\subsection{Fractions}

\subsection{Binomial coefficients}

\subsection{GCD/Euclidean Algorithm}

\end{document}
